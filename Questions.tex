1. We use EWs to compare the relative strengths of different lines in a spectrum (or the same line in different spectrums) because the continuum intensity varies. If two peaks have the same flux (area) but different continuum values, they will have different EWs (with the lower-cont value having a higher EW). The lower EW value therefore has a thinner, i.e. stronger, peak - we would not have been able to deduce this from the flux area alone.

2. These H-$\alpha$, H-$\beta$ lines etc. are the result of emission/absorption from electrons transitioning between different energy levels of (in this case) the hydrogen atom. As stars are predominantly made up of hydrogen, we use the existence of the hydrogen spectral lines to verify if the spectrum is from a star.

3. The strength of the lines is related to the surface temperature of the star. The same star at different temperatures will record a different wavelength for the location of the peak, as well as different flux (and so different EW) values. The surface temperature can be determined by Wien's law, i.e.
T = $ \frac{2.9*10^7}{\lambda_{peak}} $. Furthermore, in general, denser stars have thinner peaks due to the smaller number of collisions that happen. The effect of temperature on the EW is still dominant, however. Also, the strength of a line is proportional to the relative abundance of that element inside the star.

4. Other absorption lines would depend on the temperature and could include Titanium Oxide and Sodium for lower-temp stars or Calcium for mid-range stars or Helium for higher-temp stars. I can see strong Titanium Oxide peaks for many of the spectra and weaker Sodium ones. There may also be some ionized Calcium lines but they are not as obvious as there are, in general, 4 peaks between 3800-4000, which are very strong. In general I would say the Titanium Oxide line near 5580 is the strongest non-Hydrogen line, followed by the Calcium H and K lines.

\end{document}